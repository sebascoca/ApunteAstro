%   -------
%   Pruebas
%   -------
% Main document:  astro.tex
% bibliography:	  sec.bib
% Compilación:	  pdflatex astro.tex
%		  biber astro
%		  pdflatex astro.tex
%		  pdflatex astro.tex
%
% sino correr:	  make
% ----------------------------------------------------------------------------
% &&&&&&&&&&&&&&&&&&&&&&&&&&&&&&&&&&&&&&&&&&&&&&&&&&&&&&&&&&&&&&&&&&&&&&&&&&&&
% ----------------------------------------------------------------------------
%\documentclass[12pt,a4paper,spanish]{article}

\input{../../../tex/preambule-red}

\DeclareLanguageMapping{spanish}{spanish-apa}
%\addbibresource{bibexport.bib}
\addbibresource{sec.bib}


% comandos especiales
\newcommand{\finej}{\hspace*{\fill}$\blacklozenge$}

\hyphenation{
	}

%\newcommand\curso{5}


\title{Apunte de Astronomía}
\author{Lic. Sebastián Coca}
\date{\small v.\number\year\number\month\number\day}
% ----------------------------------------------------------------------------
% &&&&&&&&&&&&&&&&&&&&&&&&&&&&&&&&&&&&&&&&&&&&&&&&&&&&&&&&&&&&&&&&&&&&&&&&&&&&
% ----------------------------------------------------------------------------
\begin{document}
%\begin{titlepage}
%  % para generar el título del trabajo
%  añslj añsldjfa sfj\par
%  \bfseries{lalalal}\par
%  {\scshape jjjaajajaj}
%  \begin{tcolorbox}[colback=red!5!white,colframe=red!75!black] % src: overleaf
%    lala laal
% https://creativecommons.org/licenses/by-sa/4.0/
%  \end{tcolorbox}
%\end{titlepage}

\maketitle

\chapter{Astronomía de posición}

\section{Sistemas de referencia}
\subsection{Sistemas de coordenadas matemáticas}
\label{c.scm}

En los siguientes links van a poder acceder a los vídeos de las clases
preparadas para los temas desarrollados en la sección (\ref{c.scm}):

\begin{itemize}
  \item Parte 1: Sistemas de coordenadas matemáticos en 2D\\
    \href{https://youtu.be/jZOEsT9U6XA}{https://youtu.be/jZOEsT9U6XA}
  \item Parte 2: Sistemas de coordenadas matemáticos en 3D\\
    \href{https://youtu.be/wEYLrErtwt8}{https://youtu.be/wEYLrErtwt8} \\
    %\href{https://youtu.be/ENnXdLPZGPM}{https://youtu.be/ENnXdLPZGPM} \\
\end{itemize}

Más información sobre los sistemas de coordenadas pueden acceder en las
siguientes páginas de wikipedia:
\begin{itemize}
  \item 
    \href{https://es.wikipedia.org/wiki/Coordenadas\_cil\%C3\%ADndricas}
    {Sistema de coordenadas cilíndricas}
  \item  
    \href{https://es.wikipedia.org/wiki/Coordenadas\_esf\%C3\%A9ricas}
    {Sistema de coordenadas esféricas}
\end{itemize}

%https://www.lifeder.com/coordenadas-esfericas/
% https://es.wikipedia.org/wiki/Coordenadas_esf%C3%A9ricas
% https://es.wikipedia.org/wiki/Coordenadas_cil%C3%ADndricas

\subsubsection*{Actividades (sistemas de coordenadas
matemáticos)}
\small
\begin{enumerate}
  \item Transformar las siguientes coordenadas bidimensionales y graficar:
    \begin{enumerate}[a)]
	\item De rectangulares a polares \big($(x;y)\to (r;\theta)$\big):\\
	  $(3;4)$; \quad $(3;1)$; \quad $(5;5)$; \quad $(-3;-4)$; \quad $(-5;5)$;\\ 
          $(5;-5)$; \quad $(3;-1)$; \quad $(-5;-5)$; \quad $(0;-3.5)$; \quad
          $(-\sqrt{2};0)$. 
        \item De polares a rectangulares \big($(r;\theta) \to (x;y)$\big):\\
          $(2;\pi/2)$; \quad $(\sqrt{2};\pi)$; \quad $(3.5;7\pi/6)$; \quad
          $(7;-\pi/3)$; \quad $(\sqrt{1};0)$; \\
          $(\sqrt{3};\pi)$; \quad $(4;2\pi/3)$; \quad $(4;-\pi/2)$; \quad $(0;0)$;
          \quad $(\sqrt{1};-\pi)$.
    \end{enumerate}

  \item Transformar las siguientes coordenadas tridimensionales a los otros
    sistemas y graficar:
    \begin{enumerate}[a)]
      \item $(x;y;z)$: \\
	$(1;1;1)$; \quad $(-3;0;4)$; \quad $(-1;-1;-1)$; \quad $(2;2;0)$; \quad
        $(2;2;0)$; \quad $(0;2;2)$; \quad $(3;2;-1)$.
      \item $(r;\phi;z)$: \\
        $(\sqrt{2};\pi/4;0)$; \quad $(\sqrt{2};5\pi/4;0)$; \quad $(5;\pi/6;-1)$;
	\quad $(0;\pi;0)$; \quad $(3;0;0)$; \quad $(3;2\pi;0)$; \quad $(2;3\pi/2;1)$.
      \item $(\rho;\theta;\phi)$:\\
        $(\sqrt{3};\pi/4;\pi/4)$; \quad $(2;\pi;0)$; \quad $(5;0;\pi)$; \quad
        $(\sqrt{3};3\pi/4;5\pi/4)$; \quad $(2;3\pi/4;3\pi/2)$; \quad
	$(\sqrt{4};\pi/2;\pi)$; \quad $(\sqrt{9};\pi/3;7\pi/8)$.
    \end{enumerate}
\end{enumerate}
\normalsize

\subsection{Sistemas de coordenadas astronómicos}
\label{c.sca}

En los siguientes links van a poder acceder a los vídeos de las clases
preparadas para los temas desarrollados en la sección (\ref{c.sca}):

\begin{itemize}
  \item Parte 1: Sistemas de coordenadas astronómicas. Definiciones generales y
    construcción del Sistema de Coordenadas Horizontales.\\
    \href{https://youtu.be/20HwdOHqj-U}{https://youtu.be/20HwdOHqj-U}
  \item Parte 2: Sistemas de coordenadas astronómicas. Definiciones específicas
    y construcción del Sistema de Coordenadas Ecuatoriales Horarias y
    Coordenadas Ecuatoriales Absolutas.\\
    \href{https://youtu.be/TtVTJF01tew}{https://youtu.be/TtVTJF01tew} \\
  \item Parte 3: Sistemas de coordenadas astronómicas. Aplicaciones prácticas.\\
    \href{https://youtu.be/srMmv1g9UmI}{https://youtu.be/srMmv1g9UmI} \\
\end{itemize}

Más información sobre los sistemas de coordenadas astronómicos pueden acceder en
las siguientes páginas de wikipedia y en los libros recomendados:
\begin{itemize}
  \item 
    \href{https://es.wikipedia.org/w/index.php?title=Coordenadas_horizontales}
    {Coordenadas horizontales en español}
  \item 
    \href{https://en.wikipedia.org/wiki/Horizontal_coordinate_system}
    {Coordenadas horizontales en inglés}
  \item  
    \href{https://es.wikipedia.org/wiki/Coordenadas_ecuatoriales}
    {Coordenadas ecuatoriales en español}
  \item  
    \href{https://es.wikipedia.org/wiki/Coordenadas_horarias}
    {Coordenadas ecuatoriales horarias en español}
  \item 
    \href{https://en.wikipedia.org/wiki/Equatorial_coordinate_system}
    {Coordenadas ecuatoriales en inglés}
  \item 
    \href{https://es.wikipedia.org/wiki/Coordenadas_celestes}
    {Coordenadas astronómicas en español}
  \item
    \href{https://en.wikipedia.org/wiki/Celestial_coordinate_system}
    {Coordenadas astronómicas en inglés}
\end{itemize}

¡¡¡INCLUIR LOS LIBROS!!!

\subsubsection*{Actividades (coordenadas astronómicas)}
\small

\begin{enumerate}
  \item Realizar esquemas de la Esfera Celeste indicando horizonte, ecuador
    celeste, polos celestes, cenit, nadir, meridiano del lugar, primer vertical
    y puntos cardinales, tal como aparecerían para observadores situados en los
    siguientes puntos sobre la esfera terrestre:
    \begin{enumerate}[a)]
      \item sobre el Ecuador Terrestre;
      \item en el polo Sur, discuta como sería en el polo Norte;
      \item un lugar con latitud $\phi = + 30\grm$;
      \item un lugar con latitud \(\phi = - 60\grm\).
    \end{enumerate}

  \item En los esquemas del Problema anterior, señalar la trayectoria aparente
    que sigue una estrella desde su culminación superior hasta 6 horas después
    de la misma.

  \item ¿En qué parte de la esfera celeste la altura de los astros aumenta
    continuamente y en qué parte disminuye continuamente? ¿Qué sucede con el
    Ángulo Horario \(H\)?

  \item Para todos los problemas anteriores, abrir y configurar el programa
    Stellarium de tal manera de comprobar los resultados obtenidos en ellos.
    Realizar para ello capturas de pantallas ilustrativas.

  \item Determinar si la estrella es visible desde Londres bajo las siguientes
    condiciones, justificar cada caso:
    \begin{enumerate}[a)]
      \item Sus coordenadas ecuatoriales horarias son: \( (-3^h; 30\grm)\) el día
	19 de Octubre a las 22h. Determinar de manera aproximada su ascensión
	recta \(\alpha\).
      \item Sus coordenadas ecuatoriales absolutas son: \( (18^h\,32^m; -20\grm) \)
	el día 21 de Junio a media noche.
      \item Sus coordenadas horizontales son: \( (275\grm; -30\grm) \) el día 1 de
	enero a las 5h.
    \end{enumerate}
    Para los casos en los que la estrella no sea visible en ese momento,
    determinar en qué momento serán visibles.

  \item ¿Cuáles son las coordenadas ecuatoriales absolutas del Sol en los
    equinoccios y los solsticios?

  \item Calcular, de manera aproximada cuando no sea posible exacto, el acimut
    del Sol en la ciudad de Córdoba para su salida y puesta, en los equinoccios
    y los solsticios (\(\varphi_{Cba.} = -31\grm\,25'\,15''\)).

  \item En el momento de su culminación superior, determinar las distancias
    cenitales máxima y mínima del Sol para la Ciudad de Córdoba. ¿En qué época
    del año ocurren estos eventos?

  \item ¿A partir de qué latitudes será posible observar el Sol a medianoche?

  \item Determinar, para la ciudad de Córdoba, el rango de declinaciones para
    que una estrella sea circumpolar (perpetuamente visible). ¿Cuál será para
    que sea perpetuamente invisible?

  \item Para un observador situado en la ciudad de Córdoba y en el momento de
    culminación inferior del punto Aries, indicar las coordenadas horizontales y
    las ecuatoriales horarias de los siguientes puntos sobre la esfera celeste:
    \begin{enumerate}[a)]
      \item Polo Sur celeste;
      \item Punto cardinal Norte;
      \item Cenit;
      \item Punto Vernal;
      \item Punto de Libra.
    \end{enumerate}

  \item Determinar el Tiempo Sidéreo para los siguientes momentos del año:
    \begin{enumerate}
      \item 2 de enero @ 19:23$\h$.
      \item 8 de marzo @ 0:43$\h$.
      \item 12 de junio @ 17:15$\h$.
      \item 29 de octubre @ 23:50$\h$.
      \item 17 de diciembre @ 7:35$\h$.
    \end{enumerate}
    
  \item Determinar que objetos son visibles el 19 de octubre a las $23\h$. Dar
    las coordenadas horizontales aproximadas y su ángulo horario. Justificar sus
    respuestas.
    \begin{enumerate}
      \item $(\alpha;\delta)=(0^h\, 30^m;70\grm)$
      \item $(\alpha;\delta)=(0^h\, 30^m;-70\grm)$
      \item $(\alpha;\delta)=(7^h\, 58^m\, 19^s;-30\grm\, 34'\, 55'')$
      \item $(\alpha;\delta)=(19^h\, 3^m\, 19^s;-54\grm\,38')$
      \item $(\alpha;\delta)=(13^h\,1^m\,59^s; -15\grm\, 23'\, 19'')$
    \end{enumerate}

  \item Un persona observa una estrella el día 23 de septiembre a las 23$\h$ y
    registra sus coordenadas horizontales: $(A;z)=(3\grm;23.5\grm)$. Responder y
    justificar cada pregunta:
    \begin{enumerate}
      \item ¿La estrella ya realizó su culminación superior?
      \item ¿Es posible afirmar que la declinación de la estrella es positiva?
      \item Determinar su ascensión recta.
      \item Determinar sus coordenadas ecuatoriales absolutas para 15 días
	después de la observación realizada
      \item Calcular sus coordenadas horizontales 15 días antes y posteriores de
	la observación al mismo horario.
      \item Determinar hasta que fecha será visible la estrella en el mismo
	horario.
    \end{enumerate}
\end{enumerate}

\subsubsection*{Problemas de exámenes}
\begin{enumerate}
  \item Una estrella posee las siguientes coordenadas en Córdoba:
    \((A;h)=(210\grm;60\grm)\) el día 17/12 a las \(5\,h\). Determinar:
    \begin{enumerate}
      \item Tiempo Sidereo.
      \item Coordenadas Ecuatoriales absolutas y horarias.
      \item Graficar y ubicar el punto Vernal.
    \end{enumerate}

  \item Si tenemos una estrella con \( (\alpha; \delta) = (20^h\,30^{min};
    -15\grm) \) en un lugar con \( \phi = 15\grm \), y a las \( 21^h \) se ve a
    la estrella con \( (A;h)= (90\grm; 55\grm) \), responder:
    \begin{enumerate}
      \item ¿Cuánto tiempo falta para que se oculte?
      \item Calcular el TS y fecha en que es observada la estrella.
    \end{enumerate}

\end{enumerate}
\normalsize


\chapter{Movimiento de los astros}

\section{Movimiento elíptico: Leyes de Kepler}
\label{c.kepler}

En los siguientes links van a poder acceder a los vídeos de las clases
preparadas para los temas desarrollados en la sección (\ref{c.kepler}):

\begin{itemize}
  \item Movimiento de los planetas -- Ecuación de posición en coordenadas
    polares sobre una elipse.\\
    \href{https://youtu.be/c3I5bRt4-v8}{https://youtu.be/c3I5bRt4-v8}
%  \item Movimiento de los cuerpos celestes -- secciones cónicas.
%    \href{lala}{lalal}
\end{itemize}

\subsubsection*{Actividades (Leyes de Kepler)}
\small

\begin{enumerate}
  \item Luego de ver el vídeo, realizar las cuentas del minuto 14:40 donde se
    pasa de la ecuación:
    \[
      r'\,^2 = \big(r\sen(\theta)\big)^2 + \big(2ae + r \cos (\theta)\big)^2,
    \]
    a la ecuación:
    \[
      r'\,^2 = r^2 + 4ae(ae+r\cos(\theta)).
    \]
    \item Luego, utilizar la última ecuación obtenida y combinarla en conjunto
      con la ecuación de la elipse:
      \[
	r + r' = 2a,
      \]
      y llegar a la ecuación de la elipse en coordendas polares:
      \[
	r = \dfrac{a(1-e^2)}{1+e\cos(\theta)} \qquad (0\le e < 1)
      \]
    \item Calcular el radio vector $r$ para todos los planetas del Sistema Solar
      en su perihelio y afelio.
    \item Calcular la velocidades para el afelio y perihelio de los planetas del
      punto anterior con las siguientes expresiones:
      \[
	v_p^2=\dfrac{GM(1+e)}{r_p} = \dfrac{GM}{a}\left(\dfrac{1+e}{1-e}\right)
	\]
      \[
	v_a^2 = \dfrac{GM(1-e)}{r_a} =
	\dfrac{GM}{a}\left(\dfrac{1-e}{1+e}\right)
	\]
    \item Durante una observación astronómica se detecta a un NEO (Near Earth
      Object) desconocido hasta el momento. El grupo de observación desea
      catalogarlo y realizan mediciones de sus parámetros. Entre los parámetros
      que determinan, está la velocidad del mismo: \mbox{$v_{NEO}=26.2\kms$}.
      Responder:
      % a = 1.3 UA (NEO), e=0 => 26.2 km/s
      % a = 3AU; e=0.1 => vel. perihelio = 19 km/s
      % vel. de los asteroides =  10-70 km/s
      \begin{enumerate}[a)]
	\item Si el cuerpo cumple con las leyes de Kepler, ¿su velocidad
	  permanece constante o varía? Justificar.
	\item Supongamos que también nos proveen con otro dato, su excentricidad
	  $e=0$, ¿qué podemos decir acerca de la órbita del mismo y a qué tipo
	  de movimiento corresponde? Calcular su período sabiendo que su semieje
	  mayor es $a=1.3\,UA$.
	\item Ahora suponga que $e=0.1$ y que la velocidad que se midió
	  corresponde a su punto más próximo a la Tierra, determinar su periodo,
	  sus longitudes y velocidades de perigeo y apogeo, y su semieje mayor
	  y menor.
      \end{enumerate}
    \item La estrella \emph{TRAPPIST-1} es un sistema que posee 7 exoplanetas.
      La información de sus exoplanetas se puede obtener del siguiente enlace:
      \href{http://exoplanet.eu/catalog/}{http://exoplanet.eu/catalog/}.
      \begin{enumerate}
	\item Verificar si los exoplanetas corresponden a un sistema Kepleriano.
	\item Comparar las velocidades orbitales de los planetas entre sí, en el
	  periastro y apoastro y discutir sobre el resultado obtenido.
	\item Ahora considerar el sistema planetario \emph{HD 40307} y realizar
	  las mismas actividades que los puntos anteriores. Discutir sobre los
	  resultados alcanzados entre ambos sistemas.
      \end{enumerate}

\end{enumerate}

\subsubsection*{Problemas de exámenes}
\begin{enumerate}
  \item Una nave espacial cuya masa es \( m=50\kg \) orbita a Marte: \Mars, a
    una distancia de la superficie de \(1000\km\), determinar:
    \begin{enumerate}
      \item Sus energías para que sea una órbita circular. Calcular \( a_c,\,
	T,\, v\,\mathrm{ y }\, \omega \).
      \item Calcular el impulso necesario para cambiar su órbita circular a una
	elíptica con \( e=0.75 \) y semieje menor \( b=1000\km + R_\mathMars \).
	Utilizar Hohmann para el cálculo.
    \end{enumerate}

  \item La basura espacial son objetos pequeños que se mueven a altas
    velocidades. Suponga que se tiene un tornillo con \( m=50\g \) viajando a
    \( 150\kms \) a \( 1000\km \) de altura de la superficie terrestre. Se
    quiere ``limpiar'' la zona y llevar esta basura a una órbita al doble de
    alto. Responder:
    \begin{enumerate}
      \item\label{l:1} \( a_c,\, T,\, \omega\,\mathrm{ y }\, v \) que posee en su órbita.
      \item Calcular las energías previas y posteriores al cambio de órbita.
      \item Determinar los impulsos necesarios para realizar el cambio de órbita
	y el tiempo que demoran.
      \item Igual al punto \ref{l:1}) pero en la nueva órbita.
    \end{enumerate}


\end{enumerate}
\normalsize
 


% =============================================================================
% =============================================================================
% =============================================================================


\end{document}
% ---------------
\clearpage

\printbibliography
%\printbibheading

% ---------------
\clearpage

\input{anexo}

\end{document}

